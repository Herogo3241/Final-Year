\documentclass[14pt]{article}
\usepackage[utf8]{inputenc}
\usepackage[english]{babel}
\usepackage[T1]{fontenc}
\usepackage{amsmath}
\usepackage{graphicx}
\usepackage{geometry}
\geometry{margin=1in}
\usepackage{setspace}
\usepackage{ragged2e}  % For justification control

% Title and Author Info
\title{\fontsize{20pt}{24pt}\selectfont \textbf{Reinforcement Learning for ICU Treatment Planning}}

\vspace{3cm}

\author{%
Goureesh Chandra (TVE22CS069),\\ 
Ivin Mathew Kurian (TVE22CS075), \\
Muhammed Farhan (TVE22CS094), \\
Rethin Francis (LTVE22CS149) \\
\vspace{1cm}
Advisor: Prof. Divya S K \\
College of Engineering, Trivandrum, Dept. of Computer Science \& Engineering
}
\date{\today}

\begin{document}

\onehalfspacing
\maketitle
\vspace{1cm}

\renewcommand{\abstractname}{}
\begin{abstract}
\centering
\fontsize{16pt}{20pt}\selectfont \textbf{Abstract} \\
\vspace{0.5cm}
\justify
\fontsize{12pt}{15pt}\selectfont
In hospitals, doctors often face difficult decisions about how to treat critically ill patients, especially those with complex conditions like sepsis or breathing problems. This project uses artificial intelligence (AI) to help improve those decisions.

We developed a smart system that learns from real patient data collected in the Intensive Care Unit (ICU). The system uses a type of AI called reinforcement learning, which learns to make better choices over time — similar to how a human learns from experience.

Instead of testing directly on real patients, the AI first builds a simulator (or "digital twin") that can predict how a patient's health might change after receiving certain treatments, like oxygen or medication. Then, the AI experiments in this simulator to figure out the best treatment strategies.

This method helps create safer and more personalized treatment plans, which can support doctors and improve patient outcomes. It also avoids unnecessary treatments, reducing risks and costs in healthcare.
\end{abstract}

\end{document}
